
% Package imports
\usepackage{siunitx}
\usepackage{listings}
\usepackage{multirow}
\usepackage{xspace}
\usepackage{txfonts}
\usepackage{hyperref}
\usepackage[utf8]{inputenc}
\usepackage{pgfplots}
\usepackage{colortbl}
\usepackage{longtable}
\usepackage{xtab}
\usepackage{float}
\usepackage{color}
\usepackage{newfloat}
\usepackage{nicefrac}
\usepackage{datetime}
\usepackage{pdflscape}
\usepackage{multicol}
\usepackage{enumitem}
\usepackage[xspace]{ellipsis}
\usepackage{ifthen}

\DeclareFixedFont{\ttb}{T1}{txtt}{bx}{n}{8} % for bold
\DeclareFixedFont{\ttm}{T1}{txtt}{m}{n}{8}  % for normal


\newcommand{\todo}[1]{{\color{red} TODO: #1} \typeout{TODO TODO TODO #1}}
\newcommand{\todoref}[0]{{\color{red} [TODO]\xspace}\typeout{REF}}
\newcommand\nobrkhyph{\mbox{-}}


% For abbreviations such as i.e., e.g., …
% Also, omit final dot from each def.
\ExplSyntaxOn
\newcommand\latinabbrev[1]{
  \peek_meaning:NTF . {% Same as \@ifnextchar
    #1\@}%
  { \peek_catcode:NTF a {% Check whether next char has same catcode as \'a, i.e., is a letter
      #1.\@ }%
    {#1.\@}}}
\ExplSyntaxOff

\def\eg{\latinabbrev{e.g}}
\def\etal{\latinabbrev{et al}}
\def\etc{\latinabbrev{etc}}
\def\ie{\latinabbrev{i.e}}

\def\eclipsephase{Eclipse Phase\xspace}
\def\ep{EP\xspace}

\newcommand{\dice}[1]{{#1}}
\newcommand{\skill}[2][]{{\ifthenelse{\equal{#1}{}}{#2}{#2\,(\num[retain-explicit-plus]{#1})}}}
\newcommand{\modifier}[1]{{\num[retain-explicit-plus]{#1}}}
\newcommand{\dv}[1]{{DV\,#1}}


% For the \SI command
\DeclareSIUnit\years{yrs}
\DeclareSIUnit\pt{pt}
\DeclareSIUnit\px{px}
\DeclareSIUnit\dBA{dB(A)}
\DeclareSIUnit\cpm{char\,/\,min}
\DeclareSIUnit\wpm{WPM}
\DeclareSIUnit\bps{bit\,/\,s}
\DeclareSIUnit\bpm{bit\,/\,min}



% Regular tables
\def\starttabletwo{\begin{xtabular}{|p{.35\columnwidth}|p{.55\columnwidth}|}}
\def\stoptabletwo{\end{xtabular}}
\def\Rtwo #1|#2{ #1&#2}

\def\starttableone{\begin{xtabular}{|p{1.1\columnwidth}|}}
\def\stoptableone{\end{xtabular}}


\restylefloat{table}
\frenchspacing

\pgfplotsset{compat=1.8}
