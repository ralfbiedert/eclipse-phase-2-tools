% Package imports
\usepackage{siunitx}
\usepackage[table]{xcolor}
\usepackage{multirow}
\usepackage{multicol}
\usepackage{xspace}
\usepackage{txfonts}
\usepackage{hyperref}
\usepackage[utf8]{inputenc}
\usepackage{colortbl}
\usepackage{longtable}
\usepackage{xtab}
\usepackage{float}
\usepackage{newfloat}
\usepackage{pdflscape}
\usepackage{enumitem}
\usepackage[xspace]{ellipsis}
\usepackage{ifthen}
\usepackage{fancyhdr}
\usepackage{environ}
\usepackage{tikz}
\usepackage{eso-pic} % für AddToShipoutPicture
\usepackage{tikzpagenodes}
\usepackage{pifont}

\usetikzlibrary{calc}


% COLORS
\DefineNamedColor{named}{eclipsered}{rgb}{0.686,0.066,0.082}
\definecolor{tablecolor}{named}{eclipsered}


% CUSTOM COMMANDS AND ABBREVIATIONS
\newcommand{\todo}[1]{{\color{red} TODO: #1} \typeout{TODO TODO TODO #1}}
\newcommand\nobrkhyph{\mbox{-}}

\newcommand{\dice}[1]{{#1}}
\newcommand{\skill}[2][]{{\ifthenelse{\equal{#1}{}}{#2}{#2\,(\num[retain-explicit-plus]{#1})}}}
\newcommand{\modifier}[1]{{\num[retain-explicit-plus]{#1}}}
\newcommand{\dv}[1]{{DV\,#1}}

\def\eclipsephase{Eclipse Phase\xspace}
\def\ep{EP\xspace}




% For abbreviations such as i.e., e.g., …
% Also, omit final dot from each def.
\ExplSyntaxOn
\newcommand\latinabbrev[1]{
  \peek_meaning:NTF . {% Same as \@ifnextchar
    #1\@}%
  { \peek_catcode:NTF a {% Check whether next char has same catcode as \'a, i.e., is a letter
      #1.\@ }%
    {#1.\@}}}
\ExplSyntaxOff

\def\eg{\latinabbrev{e.g}}
\def\etal{\latinabbrev{et al}}
\def\etc{\latinabbrev{etc}}
\def\ie{\latinabbrev{i.e}}




% TABLES
\newsavebox{\tablebox}

% Base fancy Eclipse Phase box
% https://tex.stackexchange.com/questions/78692/using-an-environ-environment-with-newenvironment
\environbodyname\rndtablebody
\NewEnviron{rndtable}[1]{%

    % Alternating background rows
    \rowcolors{2}{tablecolor!10}{tablecolor!30}%

    % Actual table content
    % WTF DO I NEED '%' at the end of these lines since otherwise it will look ugly?
    \savebox{\tablebox}{%
        \begin{tabular}[htbp]{#1}%
            \rndtablebody%
        \end{tabular}%
    }%

    % Fancy frame
    \begin{tikzpicture}
        \begin{scope}
          \clip[rounded corners=1ex] (0,-\dp\tablebox) -- (\wd\tablebox,-\dp\tablebox) -- (\wd\tablebox,\ht\tablebox) -- (0,\ht\tablebox) -- cycle;
          \node at (0,-\dp\tablebox) [anchor=south west,inner sep=0pt]{\usebox{\tablebox}};
        \end{scope}
        \draw[rounded corners=1ex] (0,-\dp\tablebox) -- (\wd\tablebox,-\dp\tablebox) -- (\wd\tablebox,\ht\tablebox) -- (0,\ht\tablebox) -- cycle;
    \end{tikzpicture}
}


% Table with one column
\environbodyname\tableonebody
\NewEnviron{tableone}[1]{
    \begin{rndtable}{p{1.3\columnwidth} }
        \multicolumn{1}{c}{
            \cellcolor{tablecolor}\color{white} #1
    } \\
    \hline
    \tableonebody
    \end{rndtable}
}

% Table with two columns
\environbodyname\tabletwobody
\NewEnviron{tabletwo}[1]{
    \begin{rndtable}{l | p{1.1\columnwidth} }
        \multicolumn{2}{c}{
            \cellcolor{tablecolor}\color{white} #1
    } \\
    \hline
    \tabletwobody
    \end{rndtable}
}

\def\starttabletwo{\begin{xtabular}{|p{.35\columnwidth}|p{.55\columnwidth}|}}
\def\stoptabletwo{\end{xtabular}}
\def\Rtwo #1|#2{ #1&#2}

\def\starttableone{\begin{xtabular}{|p{1.1\columnwidth}|}}
\def\stoptableone{\end{xtabular}}
